\documentclass{article}

\begin{document}

\begin{titlepage}
\author{Gennadi Heimann} 
\title{Change Log} 
\date{29.12.2017} 
\maketitle
\end{titlepage}

\section{v0.0.1}
@created on 14.11.2017\\
@finished on \\

\subsection{}
@created on     14.11.2017\\
@finished on   \\

Der naechste Schritt wird erst freigegeben, wenn alle Abhaengigkeiten 
der Komponenten geprueft sind. Beim Pruefen von der ausgewaelte 
Komponente wird auch die SelectionCriterium geprueft, wieviel Komponenten 
in dem Schritt ausgewaelt werden koennen.

Die ausgewaelte Komponente wird in der CurrentConfig gesichert, damit das
SelectionCriterium bewertet kann.

\subsection{}
@created on     4.12.2017\\
@finished on   \\

Die aktuelle Konfiguration beh\"lt gesamte Information \"uber die hinzugef\"gte
Komponente. Dazu geh\"ohren alle Abh\"ngigkeiten und Einschr\"ankungen und
Komponente mit $<exclude>$ Markierung.


\section{v0.0.2}
\subsection{}
@created on 13.12.2017\\
@finished on \\
Jede Schritt in dem Konfigurator kann eine Abh\"angigkeit zu dem Schritt oder
Komponente haben. Das bedeutet, dass geladene Schritt eine Komponente oder einen
Schritt ausschlie\ss{}en oder fordern.\\

\subsection{}
@created on 18.12.2017\\
@finished on \\

Bei dem Abschluss der Konfiguration muss auf die Konsistenz den letzten
Schritte geachtet werden. Es kann sein dass in einem Schritt paar Komponente
ohne weiteren Schritt und paar mit weiterem Schritt vorhanden. (siehe
szenario\_3)\\

\subsection{}
@created on 20.12.2017\\
@finished on \\

Bei der Auswahl der gleichen Komponente innerhalb eines Schrittes muss erkannt
werden und eine Info an den Webclient gesendet.\\

Bei der $<Dependency>$ werden die Parameter $<message>$ und $<status>$ nicht
gebraucht. Sie k\"onnen auch gel\"oscht werden.\\

\subsection{}
@created on 21.12.2017\\
@finished on \\

Zu der $<ComponentOut>$ die $<componentId>$ Parameter bei allen Status
hinzuf\"ugen.\\

\subsection{}
@created on 21.12.2017\\
@finished on \\

Abw\"ahlen der Komponente muss auch implimentiert werden. Die Komponente wird
auch asu der $<CurrentConfig>$ entfernt

\end{document}
