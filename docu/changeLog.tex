\documentclass{article}

\usepackage[utf8]{inputenc}

\begin{document}

\begin{titlepage}
\author{Gennadi Heimann} 
\title{Change Log} 
\date{29.12.2017} 
\maketitle
\end{titlepage}

\section{v0.0.1}
@created on 14.11.2017\\
@finished on \\

\subsection{}
@created on     14.11.2017\\
@finished on   21.12.2017\\

Der nächste Schritt wird erst freigegeben, wenn alle Abhängigkeiten 
der Komponenten geprüft sind. Beim Prüfen von der ausgewählte 
Komponente wird auch die SelectionCriterium geprüft, wieviel Komponenten 
in dem Schritt ausgewählt werden können.

Die ausgewählte Komponente wird in der CurrentConfig gesichert, damit das
SelectionCriterium bewertet weden kann.\\

\section{v0.0.2}

\subsection{Requirements}

\subsubsection{}
@created on     4.12.2017\\
@finished on   19.03.2018\\

Die aktuelle Konfiguration behält gesamte Information über die hinzugefügte
Komponente. Dazu gehören alle Abhängigkeiten und Einschränkungen und
Komponente mit $<exclude>$ Markierung.\\

\subsubsection{}
@created on 18.12.2017\\
@finished on 19.03.2018\\

Bei dem Abschluss der Konfiguration muss auf die Konsistenz den letzten
Schritte geachtet werden. Es kann sein, dass in einem Schritt paar Komponente
ohne weiteren Schritt und paar mit weiterem Schritt haben. (siehe
szenario\_3)\\

\subsubsection{}
@created on 20.12.2017\\
@finished on 19.03.2018\\

Bei der Auswahl der gleichen Komponente innerhalb eines Schrittes muss 
diese Änderung erkannt werden und eine Info an den Webclient gesendet.\\

\subsubsection{}
@created on 21.12.2017\\
@finished on 21.12.2017\\

Zu der $<ComponentOut>$ die $<componentId>$ und $<stepId>$ Parameter bei
allen Status hinzuf\"ugen.\\

\subsubsection{}
@created on 21.12.2017\\
@finished on 19.03.2018\\

Abw\"ahlen der Komponente muss auch implimentiert werden. Die Komponente wird
auch aus der $<CurrentConfig>$ entfernt.\\

\subsection{Test}
\begin{tabbing}

\noindent Scenario 002-1: \hspace*{0,2cm} \= S1 $->$ C11, C12 $|$ S2 $->$ C21
$|$ S3 $->$ C31 $|$ S2 $->$ C21 ok \\

\noindent Scenario 002-2: \>  S1 $->$ C11, C11 ok \\

\noindent Scenario 002-3: \>  S1 $->$ C11, C12, C11 ok \\

\noindent Scenario 002-4: \>  S1 $->$ C11, C12, C12 ok \\

\noindent Scenario 002-5: \>  S1 $->$ C11 $|$ S2 $->$ C21 $|$ S3 $->$ C31 ok \\

\noindent Scenario 002-6: \>  S1 $->$ C11 ok \\

\noindent Scenario 002-7: \>  S1 $->$ C11, C12 ok \\

\noindent Scenario 002-8: \>  S1 $->$ C11, C12, C13 ok \\

\noindent Scenario 002-9: \>  S1 $->$ C11, C12, C13, C11 ok \\

\noindent Scenario 002-10: \> S1 $->$ C11, C12, C13, C11, C12 ok \\

\noindent Scenario 002-11: \> S1 $->$ C11, C12, C13, C11, C12, C13 ok \\

\noindent Scenario 002-12: \> S1 $->$ C11, C12, C13, C11, C12, C13, C11 ok \\

\noindent Scenario 002-13: \> S1 $->$ C11, C12, C13, C11, C12, C13, C11,
C12 ok \\

\noindent Scenario 002-14: \> S1 $->$ C11, C12, C13, C11, C12, C13, C11, C12,
C13 ok \\ 

\noindent Scenario 002-15: \> S1 $->$ C11, C12 $|$ S2 $->$ C21 ok \\

\noindent Scenario 002-16: \> S1 $->$ C11, C12 $|$ S2 $->$ C21, C22\\

\end{tabbing}

\subsubsection{Szenario v002-1}

configUrl=http://contig1/user29\_v016\\

Diese Szenario nur m\"oglich bei der Einstellung von
$<configurationCourse=sequence>$.\\

\noindent Es gibt folgende Verläfe: $<sequence>, <sabstitute>$. 
Diese Verläufe beziehen sich in großen Ganzen nur auf der Verhalten des Clients des Konfigurators.
Bei der $<sequence>$ werden die nächste Schritte an der vorherige Schritt angehängt, 
so dass der Benutzer alle abgearbeitete Schritte sieht. Bei der $<sabstitute>$ der 
nächste Schritt ersetzt den vorherigen. Der Server benötigt die Information ob der 
vorherige Schritte durch der Auswahl des Benutzers (z.B. Button vorherige Schritt) geladen 
werden soll ($<sabstitute>$) oder der Benutzer die Änderung in den vorherigen Schritten 
sofort vornehmen kann, da er visualisiert ist ($<sequence>$). 

\begin{enumerate}
  \item Server: Start Konfiguration $->$ Schritt 1
  \item Client: Auswahl $->$ Komponent 1-1
  \item Server: Komponent 1-1 hinzugef\"ugt, aktuelle Konfiguration angepasst
  \item Client: Auswahl $->$ Komponent 1-2
  \item Server: Komponent 1-2 hinzugef\"ugt, aktuelle Konfiguration angepasst
  \item Client: Auswahl: $->$ n\"achste Schritt laden
  \item Server: Lade Schritt 2
  \item Client: Auswahl $->$ Komponent 2-1
  \item Server: Komponent 2-1 hinzugef\"ugt, aktuelle Konfiguration angepasst
  \item Client: Auswahl: $->$ n\"achste Schritt laden
  \item Server: Lade Schritt 3
  \item Client: Auswahl: $->$ Komponent 3-1
  \item Server: Komponent 3-1 hinzugef\"ugt, aktuelle Konfiguration angepasst
  \item Client: Auswahl $->$ Komponent 2-1
  \item Server: Warnung: Komponente 2-1 und Komponente 3-1 werden aus der
  Konfiguration entfernt.
  \item Client: Auswahl $->$ Ja
  \item Server: Komponente 2-1, Schritt 3 mit Komponente 3-1 aus der aktuellen
  Konfiguration entfernt.
\end{enumerate}

\subsubsection{Szenario 002-2}

configUrl=http://contig1/user29\_v016\\

Globale Einstellungen $<configurationCourse=sequence>$

\begin{enumerate}
  \item Server: Start Konfiguration $->$ Schritt 1
  \item Client: Auswahl $->$ Komponent 1-1
  \item Server: Komponent 1-1 wurde in die aktuelle Konfiguration hinzugef\"ugt
  \item Client: Auswahl $->$ Komponent 1-1
  \item Server: Komponent 1-1 wurde aus der aktuellen Konfiguratiuon entfernt
\end{enumerate}

\subsubsection{Szenario 002-3}

configUrl=http://contig1/user29\_v016\\

Globale Einstellungen $<configurationCourse=sequence>$

\begin{enumerate}
  \item Server: Start Konfiguration $->$ Schritt 1
  \item Client: Auswahl $->$ Komponent 1-1
  \item Server: Komponent 1-1 wurde in die aktuelle Konfiguration hinzugef\"ugt
  \item Client: Auswahl $->$ Komponent 1-2
  \item Server: Komponent 1-2 wurde in die aktuelle Konfiguration hinzugef\"ugt
  \item Client: Auswahl $->$ Komponent 1-1
  \item Server: Komponent 1-1 wurde aus der aktuellen Konfiguratiuon entfernt
\end{enumerate}

\subsubsection{Szenario 002-4}

configUrl=http://contig1/user29\_v016\\

Globale Einstellungen $<configurationCourse=sequence>$

\begin{enumerate}
  \item Server: Start Konfiguration $->$ Schritt 1
  \item Client: Auswahl $->$ Komponent 1-1
  \item Server: Komponent 1-1 wurde in die aktuelle Konfiguration hinzugef\"ugt
  \item Client: Auswahl $->$ Komponent 1-2
  \item Server: Komponent 1-2 wurde in die aktuelle Konfiguration hinzugef\"ugt
  \item Client: Auswahl $->$ Komponent 1-2
  \item Server: Komponent 1-2 wurde aus der aktuellen Konfiguratiuon entfernt
\end{enumerate}

\subsubsection{Szenario 002-5}

configUrl=http://contig1/user29\_v016\\

Globale Einstellungen $<configurationCourse=sequence>$

\begin{enumerate}
  \item Server: Start Konfiguration $->$ Schritt 1
  \item Client: Auswahl $->$ Komponent 1-1
  \item Server: Komponent 1-1 wurde in die aktuelle Konfiguration hinzugef\"ugt
  \item Client: Auswahl: $->$ n\"achste Schritt laden
  \item Server: Lade Schritt 2
  \item Client: Auswahl $->$ Komponent 2-1
  \item Server: Komponent 2-1 wurde in die aktuelle Konfiguration hinzugef\"ugt
  \item Client: Auswahl: $->$ n\"achste Schritt laden
  \item Server: Lade Schritt 3
  \item Client: Auswahl $->$ Komponent 3-1
  \item Server: Komponent 3-1 wurde in die aktuelle Konfiguration hinzugef\"ugt
  \item Client: Auswahl: $->$ n\"achste Schritt laden
  \item Server: Error -> Es existiert keiner weiteren Schritt
\end{enumerate}

\subsubsection{Szenario 002-6}

configUrl=http://contig1/user29\_v016\\

Schritt 1 $->$ Component 1\\

Status:

\begin{itemize}
  \item ComponentType $->$ DefaultComponent
  \item SelectedComponent $->$ AddedComponent
  \item SelectionCriterium $->$ AllowNextComponent
  \item ExcludeDependency $->$ NotExcludeComponent
  \item Common $->$ Success
\end{itemize}

\subsubsection{Szenario 002-7}

configUrl=http://contig1/user29\_v016\\

\noindent Schritt 1 $->$ Component 1, Component 2\\

\noindent Status:

\begin{itemize}
  \item ComponentType $->$ DefaultComponent
  \item SelectedComponent $->$ AddedComponent
  \item SelectionCriterium $->$ RequireNextStep
  \item ExcludeDependency $->$ NotExcludeComponent
  \item Common $->$ Success
\end{itemize}

\subsubsection{Szenario 002-8}

configUrl=http://contig1/user29\_v016\\

\noindent Schritt 1 $->$ Component 1, Component 2, Component 3\\

\noindent Status:

\begin{itemize}
  \item ComponentType $->$ DefaultComponent
  \item SelectedComponent $->$ NotAllowedComponent
  \item SelectionCriterium $->$ RequireNextStep
  \item ExcludeDependency $->$ ExcludeComponent
  \item Common $->$ Success
\end{itemize}

\subsubsection{Szenario 002-9}

configUrl=http://contig1/user29\_v016\\

\noindent Schritt 1 $->$ Component 1, Component 2, Component 3, Component 1\\

\noindent Status:

\begin{itemize}
  \item ComponentType $->$ DefaultComponent
  \item SelectedComponent $->$ RemovedComponent
  \item SelectionCriterium $->$ AllowNextComponent
  \item ExcludeDependency $->$ NotExcludedComponent
  \item Common $->$ Success
\end{itemize}

\subsubsection{Szenario 002-10}

configUrl=http://contig1/user29\_v016\\

\noindent Schritt 1 $->$ Component 1, Component 2, Component 3, Component 1,
Component 2\\

\noindent Status:

\begin{itemize}
  \item ComponentType $->$ DefaultComponent
  \item SelectedComponent $->$ RemovedComponent
  \item SelectionCriterium $->$ RequireComponent
  \item ExcludeDependency $->$ NotExcludeComponent
  \item Common $->$ Success
\end{itemize}

\subsubsection{Szenario 002-11}

configUrl=http://contig1/user29\_v016\\

\noindent Schritt 1 $->$ Component 1, Component 2, Component 3, Component 1,
Component 2, Component 3\\

\noindent Status:

\begin{itemize}
  \item ComponentType $->$ DefaultComponent
  \item SelectedComponent $->$ AddedComponent
  \item SelectionCriterium $->$ RequireNextStep
  \item ExcludeDependency $->$ NotExcludedComponent
  \item Common $->$ Success
\end{itemize}

\subsubsection{Szenario 002-12}

configUrl=http://contig1/user29\_v016\\

\noindent Schritt 1 $->$ Component 1, Component 2, Component 3, Component 1,
Component 2, Component 3, Component 1\\

\noindent Status:

\begin{itemize}
  \item ComponentType $->$ DefaultComponent
  \item SelectedComponent $->$ NotAllowedComponent
  \item SelectionCriterium $->$ RequireNextStep
  \item ExcludeDependency $->$ ExcludeComponent
  \item Common $->$ Success
\end{itemize}

\subsubsection{Szenario 002-13}

configUrl=http://contig1/user29\_v016\\

\noindent Schritt 1 $->$ Component 1, Component 2, Component 3, Component 1,
Component 2, Component 3, Component 1, Component 2\\

\noindent Status:

\begin{itemize}
  \item ComponentType $->$ DefaultComponent
  \item SelectedComponent $->$ NotAllowedComponent
  \item SelectionCriterium $->$ RequireNextStep
  \item ExcludeDependency $->$ ExcludeComponent
  \item Common $->$ Success
\end{itemize}

\subsubsection{Szenario 002-8}

configUrl=http://contig1/user29\_v016\\

\noindent Schritt 1 $->$ Component 1, Component 2, Component 3, Component 1,
Component 2, Component 3, Component 1, Component 2, Component 3\\

\noindent Status:

\begin{itemize}
  \item ComponentType $->$ DefaultComponent
  \item SelectedComponent $->$ RemovedComponent
  \item SelectionCriterium $->$ RequireComponent
  \item ExcludeDependency $->$ NotExcludeComponent
  \item Common $->$ Success
\end{itemize}

\section{v0.0.3}

\subsection{Requirements}

\subsubsection{}
@created on 13.12.2017\\
@finished on \\
Jede Schritt in dem Konfigurator kann eine Abh\"angigkeit zu dem Schritt oder
Komponente haben. Das bedeutet, dass geladene Schritt eine Komponente oder einen
Schritt ausschlie\ss{}en oder fordern.\\

\subsubsection{}
@created on 21.12.2017\\
@finished on \\

Bei der Aufruf von $<NextStepIn>$ der Web-Client braucht die ID vom vorherigen
Schritt. Damit diese beim Bedarf vor dem neuen Schritt gel\"oscht werden kann.

\subsubsection{}
@created on 21.12.2017\\
@finished on \\

Einf\"uhrung von ODBOName. Es bei jeder Aktion nur einmal die Daten aus der DB
geholt und in eien ODBOName Objekt gemapt. Die Aktion arbeitet nur mit diesem
Objekt ohne zus\"atzlich in der DB abzufragen.

\subsubsection{}
@created on 13.02.2018\\
@finished on \\

In der ConfigVertex sollen die Konfigurationen von dem Datenbank einegegeben.
Bei dem Laden von ersten Schritt werden diese Konfigurationen geladen.

\end{document}

























