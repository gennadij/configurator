\documentclass{article}
\usepackage[utf8]{inputenc}

\begin{document}

\begin{titlepage}

\author{Gennadi Heimann} 
\title{Generischer Konfigurator} 
\date{29.12.2017} 
\maketitle
\end{titlepage}

\section{Logik Beschreibung}

Die Abhängigkeiten im Konfigurator bilden die Regel. Jede Komponente hat eine
Ausgangsabhängigket (outDependency) und eine Eingangsabhängigket
(inDependency). Die Ausgangsabhängigkeit bewirkt auf ihre Komponente mit der
Regel die in der Abhngigkeit hintergelegt ist. 
Es gibt zwei Arten (dependencyType) von Abhängigkeiten $<exclude>$ und
$<require>$. Die $<exclude>$ Abhngigkeit schlie\ss{}t die Komponente aus der
Konfiguration aus. Die $<require>$ Abhängigkeit fordert wiederum, dass die
Komponente in die Konfiguration hinzugefügt werden muss. Das Verhalten der
Abhängigkeiten wird von Parameter $<visualization>$ gesteuert. Der Benutzer
des Konfigurators wählt die Komponente innerhalb eines Schrittes und die
Logik das Konfigurators prüft die Abhängigkeiten und anhand der Parameter
wird weitere Verlauf gesteuert. In der darauffolgende Beschreibung werden
einzelne Parameter erklärt.

\begin{itemize} 
    \item visualization bei der $<exclude>$ Komponenten
    \begin{itemize}
        \item auto 
        \begin{itemize}
            \item Die Komponente wird automatisch aus der Konfiguration
            entfernt.Wenn die ausgeschlossene Komponente weiter auf
            eine $<exclude>$ Komponente verweißt, werden diese Komponente 
            recursive ausgeschlossen. Das 
            bedeutet, dass bei der späterem Verlauf werden diese
            Komponente nicht gezeigt. Bei der $<require>$ Komponenten
            wird dieses Vorgehen nicht angewendet. Die Komponente wird nicht aus der 
            Konfiguration entfernt.
        \end{itemize}
        \item marked + selectable
        \begin{itemize}
            \item Die IN-Komponente wird visual merkiert, dass sie
            $<exclude>$ Komponente ist und der Benutzer hat die
            Möglichkeit diese Komponente auszuwählen.
            Wenn der Benutzer wählt diese Komponente aus, wird ein
            Hinweis dem Benutzer vorgeschlagen, wie er die Abhängigkeiten
            in seiner Konfiguration lösen kann. Es wird eine andere
            OUT-Komponente aus dem gleichem Schritt vorgeschlagen. Wenn in
            diesem Schritt keinen Vorschlag gemacht werden kann wird weiter
            in dem übergeordneten Schritt weiter gesucht. Bei jedem
            vorgeschlagenen Komponente wird weiterhin nach Abhängigkeiten
            geprüft.
        \end{itemize}
        \item marked + unselectable
        \begin{itemize}
            \item Die Komponente wird visuel markiert aber kann nicht
            ausgewählt werden. Die Abhängigkeiten werden gleich wie bei der
            Visualizationparameter $<auto>$ behandelt.
        \end{itemize}
    \end{itemize}
    \item require
    \begin{itemize}
        \item auto
        \begin{itemize}
          \item Die Komponente wird automatisch in die Konfiguration hinzugefügt.
          Es wird geprüft, ob der IN-Komponente der Abhängigkeit weitere
          $<require>$ oder $<exclude>$ Komponente hat. Wenn ja, wird gemäß der
          Parameter in der Abhängigkeit die Konfiguration angepasst. Wenn die
          IN-Koponente über mehrere Schritte in der Konfiguration entfernt ist. Der
          Konfigurator wird, dann in Laufe der Konfiguration bei jedem Schritt dem
          Benutzer einen Hinweis geben, welche Komponente der Benutzer auswählen
          kann/muss um zu dem Schritt mit der $<require>$ Komponente zu kommen. Die
          Komponente wird visual markiert, dass sie schon zu der Konfiguration
          hinzugefügt wurde. Diese Komponente ist auch $<unselectable>$. Bei der
          Abhängigkeit in gleichem Schritt wird die Komponente ohne jegliche
          Hinweise zu der Konfiguration hinzugefügt und dementsprechend markiert.
        \end{itemize}
    
        \item selectable
    
        \begin{itemize}
            \item Bevor die Komponente zu der Konfiguration hinzugefügt wird, wird
            dem Benutzer einen Hinweis mit dem Auswahl gegeben. Der Benutzer kann
            entscheiden ob er die $<required>$ Komponente zu der Konfiguration
            hinzufügen will.
        \end{itemize}
    \end{itemize}
\end{itemize}

Die aktuelle Konfiguration behält die gesamte Information (Abhngigkeiten)
von der hinzugef\"ugten Komponente. Diese Information hilft dem Konfigurator die
Pr\"ufungen durchf\"uhren, wenn der Web-Client Fehler macht. Zum Beispiel, wenn
der Web-Client dem Benutzer eine Komponente auszuwählen erlaubt, die von
Server als $<unselecteble>$ oder $<remove>$ markiert wurde. In diesem Fall
bekommt der Benutzer die Fehlermeldung.

\subsection{Auswahlkriterium}

Die Anzahl der Komponente die innerhalb eines Schrittes ausgewählt werden
können, werden über den Parameter $<SelectionCriterum>$ gesteuert. Dieses
Kriterium definiert für jeden Schritt die maximale und minimale Anzahl der
Komponente, die von dem Benutzer ausgewählt werden können. Das Auswahlkriterium 
wird nach der Prüfung der Abhängigkeiten durchgeführt. Es werden folgende Fälle
unterschieden:

\begin{itemize}
    \item RequireComponent 
    \begin{itemize}
        \item Wenn die Anzahl der ausgewählten Komponente kleiner als minimale
        Kriterium. Der Benutzer wird aufgefordert weiter Komponente
        auszuwhlen, bis minimale Kriterium erreicht wird.\\
        \textbf{min $>$ countOfComponents}
    \end{itemize}
    \item RequireNextStep
        \begin{itemize}
            \item Wenn alle Komponente in dem Schritt ausgewählt sind, wird
            der Konfigurator zu dem nächstem Schritt führen.\\
            \textbf{max == countOfComponents}\\
            Wenn der minimale Kriterium gleich Null und maximale Kriterium
            größer 1 ist, kann der Benutzer ohne den Auswahl der
            Komponente zu dem nächsten Schritt wechseln.\\
            \textbf{min == 0 and max $>$ 1} ist nicht erlaubt\\
            \textbf{min == 0 and max == 0} ist nicht erlaubt.
        \end{itemize}
    \item AllowNextComponent
    \begin{itemize}
        \item  Wenn dem Benutzer noch weitere Komponente auszuwählen erlaubt.
        In diesem Fall kann der Benutzer weitere Komponente auszuwählen oder zu
        dem nächstem Schritt wechseln.\\
        \textbf{min $<=$ countOfComponents and\\ max $>$ countOfComponents}
    \end{itemize}
\item ExcludeComponent
    \begin{itemize}
        \item Wenn der Benutzer mehr Komponente als erlaubt auswählen
        möchte.\\
        \textbf{max $<$ countOfComponents}
    \end{itemize}
\end{itemize}

\noindent\textbf{$<$countOfComponents$>$} = Anzahl der vorher
ausgewählten Komponenten.
Die Anzahl der Komponenten wird aus der aktuellen Konfiguration ermittelt\\\\

\noindent Die Variable $<countOfComponents>$ wird vor der Prüfung auf die
Auswahlkriterium immer angepasst. Die Ausgewählte Komponente wird zuerst mit aktuellen
Konfiguration verglichen. \\\\
Wenn die ausgewählte Komponente nicht in der
bestehenden Konfiguration existiert, wird $<countOfComponents>$ auf 1
inkrementiert. Somit wird temporär die ausgewählte Komponente zu der
aktuellen Konfiguration hinzugefügt. Nach dem alle Prüfungen erfolgreich abgelaufen wird
diese Komponente zu der aktuellen Konfiguration hinzugefügt.\\\\
Wenn die ausgewählte Komponente in der aktuellen Konfiguration gefunden wird.
Das bedeutet, dass die Komponente abgewählt war und aus der Konfiguration
entfernt werden muss, deswegen wird $<countOfComponents>$ nicht verändert.

Änderung 11.03.2018 bei der RemovedComponent wird selectedComponent - 1
ausgeführt.

\subsection{Hinweisfeld zu jeder Aktion}
Der Server liefert bei jedem Aktion die Erklärungen zu diesen Aktion, deswegen
bei dem Webclient sollte einen Bereich geben wo diese Information Präsentiert
werden kann.

%\subsection{Auswahl der Komponente}

%Der Server bekommt von dem Web-Client die ausgewählte Komponente.

%\begin{enumerate}
%  \item Die ausgew\"ahlte Komponente wird mit der aktuelle Konfiguration
%  vergliechen. \\
%  Wenn ausgew\"ahlte Komponente in aktueller Konfiguration exestiert,
%  \begin{itemize}
%    \item JA $->$ wird die Komponente auf die $<require>$ Abh\"angigkeit
%%    gepr\"uft.
%    Wenn andere Komponente die ausgew\"ahlte komponente ben\"otigt,
%    \begin{itemize}
%      \item JA $->$ wird die Regel, die in der Abh\"angigkeit steht angewendet.
%      \item NEIN $->$ wird die Komponente aus der aktuellen Konfiguration
%      entfernt und sind keine weitere Pr\"ufungen notwendig. Bei dieser
%      Pr\"ufung m\"ussen die Komponente und die Schritte auf Konsistenz
%      prufen. Die Komponente wird als \\REMOVED\_COMPONENT markiert
%    \end{itemize}
 %   \item NEIN $->$ wird die Komponente als ADDED\_COMPONENT markiert und
%    an weiteren Pr\"ufungen weitergegeben.
%  \end{itemize} 
%  \item Unabh\"angig von der aktuellen Konfiguration wird die ausgew\"ahlte
%  Komponente auf <$exclude>$ und $<require>$ Abh\"angigkeiten gepr\"uft.
%  \begin{itemize}
%    \item die Komponente hat eine $<require>$ Eingangsabh\"angigkeit $->$ Die
%    abh\"angige Komponente werden für weieter Prüfung ermittelt. Der Status wird
%    als REQUIRE\_COMPONENT.
%    \item die Komponente hat keine $<require>$ Eingangsabh\"angigkeit $->$ Die
%    ausgew\"ahlte Komponente wird ohne weiter Aktionen auszuf\"uhren als
%    NOT\_REQUIRE\_COMPONENT.
%    \item die Komponente hat eine $<exclude>$ Eingangsabh\"angigkeit $->$ Die
%    ausgew\"ahlte Komponente wird nach der Regeln in der Abh\"angigkeit
%    entweder trotz der Ausschluss zu der aktuellen Konfiguration hinzugef\"ugt
%    oder nicht zu der aktuellen Konfiguration hinzugef\"ugt. Der Status wird
%    als EXCLUDE\_COMPONENT markiert.
%    \item die Komponente hat keine $<exclude>$ Eingangsabh\"angigkeit $->$ Die
%    ausgew\"ahlte Komponente wird ohne weiter Aktionen auszuf\"uhren als
%    NOT\_EXCLUDE\_COMPONENT.
%%  \end{itemize}
%%  \item Die Status von der Vergleich in der aktuellen Konfiguration und von der
%  Pr\"ufung der Abh\"angigkeiten wird weiter zu der Pr\"ufung der
%  Auswahlkriterium weitergegeben.
%  \begin{itemize}
%    \item ADDED\_COMPONENT $->$
%    \begin{itemize}
%      \item REQUIRE\_COMPONENT $->$ Die abh\"angige Komponente werden je nach
%      Regel in der Abh\"angigkeit zu der aktuellen Konfiguration hinzugef\"ugt
%      oder dem Regeln entsprechende Behandlung angestossen.\\
%      Desweiteren werden
%      alle exclude Komponente innerhalb eines Schrittes und deren Regeln
%      gepr\"uft, Wenn die Regeln schliessen die Komponente aus,
%      \begin{itemize}
%        \item JA $->$ wird die List von allen innerhalb eines
%        Schrittes $<excluded>$ Komponenten, mit der Liste allen nicht ausgew\"ahlten
%        Komponenten verglichen. Wenn beide Listen gleich sind,
%        \begin{itemize}
%          \item JA $->$ wird der Status REQUIRE\_NEXT\_STEP gesetzt. Dadurch
%          wird die Anforderung zu dem n\"achstem Schritt bei dem Auswahl der
%          Komponente, der alle andere Komponente innerhab des Schrittes
%          ausschliesst, erkannt.
%          \item NEIN $->$ wird eine standartmässige \"Uberpr\"ufung der
%          Auswahlkriterium gemacht. (siehe Auswahlkriterium)
%        \end{itemize}
%        \item NEIN $->$ Definition folgt
%      \end{itemize}
%      \item NOT\_REQUIRE\_COMPONENT $->$ Es werden
%      alle exclude Komponente innerhalb eines Schrittes und deren Regeln
%      gepr\"uft, Wenn die Regeln schliessen die Komponente aus,
%      \begin{itemize}
%        \item JA $->$ wird die List von allen innerhalb eines
%%        Schrittes $<excluded>$ Komponenten, mit der Liste allen nicht ausgew\"ahlten
%        Komponenten verglichen. Wenn beide Listen gleich sind,
%        \begin{itemize}
%          \item JA $->$ wird der Status REQUIRE\_NEXT\_STEP gesetzt. Dadurch
%          wird die Anforderung zu dem n\"achstem Schritt bei dem Auswahl der
%          Komponente, der alle andere Komponente innerhab des Schrittes
%          ausschliesst, erkannt.
%          \item NEIN $->$ wird eine standartmässige \"Uberpr\"ufung der
%          Auswahlkriterium gemacht. (siehe Auswahlkriterium)
%%        \end{itemize}
%        \item NEIN $->$ Definition folgt
%      \end{itemize}
%      \item EXCLUDE\_COMPONENT $->$ setzen der Status bei der Vergleich der
%      aktuellen Konfiguration zu der NOT\_ALLOWED\_COMPONENT, wenn der Regel in
%      der Abh\"angigkeit das erlaubt. Die Komponente wird nicht zu der aktuellen
%      Konfiguration hinzuge\"ugt. Der Staus der Auswahlkriterium wird auf
%      EXCLUDED\_COMPONENT gesetzt.
%      \item NOT\_EXCLUDE\_COMPONENT $->$ 
%    \end{itemize}
%    \item REMOVED\_COMPONENT $->$ SelectionCriterium muss geprueft werden.
%%%  \end{itemize}
%    %\item Die Komponente $<Vertex>$ aus der DB lesen. $<vComponent>$
    
    %\item Alle ausgehende und eingehende Abhngigkeiten $<Edge>$ dieser
%     Komponente aus DB lesen.
    
%     \item Die Abhngigkeiten werden auf folgende Arten gefiltert:
%     
%     \begin{itemize}
%         \item Die $<exclude>$ ausgehende Abhngigkeiten
%         $<excludeDependenciesOut>$
%         \item Die $<exclude>$ eingehende Abhngigkeiten
%         $<excludeDependenciesIn>$
%         \item Die $<require>$ ausgehende Abhngigkeiten
%         $<requireDependenciesOut>$
%         \item Die $<require>$ eingehende Abhngigkeiten
%         $<requireDependenciesIn>$
%     \end{itemize}
%     
%     \item Zu jede Abhngigkeit werden dazugeh\"orige abh\"qangige Komponente
%     ermittelt.
%     
%     \item Der Vaterschritt der ausgewhlten Komponente wird aus der DB
%     abgefragt. $<vFatherStep>$
%     
%     \item Aus diesem Vaterschritt wird das Auswahlkriterium ausgelesen.
%     $<selectionCriterium>$
%     
%     \item Der aktuelle Schritt wird aus der Aktuellen Konfiguration ebenso
%     ausgelesen. $<currentStep>$
%     
%     \item Im aktuellem Schritt aus der aktuellen Konfiguration wird gepr\"uft,
%     ob die ausgewhlte Komponnte in vorherigen Aktionen ausgewhlt war.
%     Ausf\"uhrende Funktion $<checkSelectedComponent(currentStep,componentId)>$
%     
%     \begin{itemize}
%         \item JA $->$ Die Komponente aus der aktuellen Konfiguration wird
%         gel\"oscht. Der Status ist $<RemovedComponent>$
%         Wenn die zu l\"oschende Komponente die Abhngigkeiten hat, werden
%         diese dementsprechend der Regel behandelt.
%         \item NEIN $->$ Der Status ist $<AddedComponent>$
%     \end{itemize}
%     
%     \item Nachdem die aktuelle Konfiguration konsistent ist, werden
%     alle vorher ausgewhlte Komponente des Schrittes aus der
%     aktuellen Konfiguration ausgelesen. $<previousSelectedComponents>$
%     
%     \item Die definierte Auswahlkriterium im Vaterschritt wird mit dem vorher
%     ausgewhlten Komponenten verglichen und nach der Regeln der
%     Auswahlkriterium wird die aktuell ausgewhlte Komponente entweder zu der
%     aktuellen Konfiguration hizugef\"ugt oder nicht hinzugef\"ugt.
%     Ausf\"uhrende Funktion $<checkSelectionCriterium(previousSelectedComponents.size, selectionCriterium, statusSelectedComponent)>$
%     
%     \item Danach wird geschaut ob die ausgewhlte KOmponente von einem anderen
%     Komponente ausgeschlossen wird.
%     Ausf\"uhrende Funktion $<checkExcludeDependencies(currentStep.get, excludeDependenciesIn)>$
%     
%     \item Als nchstes wird geschaut ob die Komponente nchsten Schritt hat.
%     Ausf\"uhrende Funktion $<checkNextStepExistence(vComponent)>$
%     
%     \item Der allgemeiner Status wird zurzeit auf erfolgreich gesetzt.
%     
%     \item Alle erreichen Status werden zu der $<StatusComponent>$ hinzugef\"ugt.
%     
%     \item Alle Abhengigkeiten von $<requireDependenciesOut>$ und
%     $<excludeDependenciesOut>$ zu eier Liste hinzugef\"ugt.
%     
%     \item Danach wird mit Abhngigkeit des Statuses die $<ComponentOut>$
%     zusammengesetzt
    
%\end{enumerate}

\subsection{Status}

\subsubsection{StartConfig}

\begin{itemize}
	\item FIRST\_STEP\_NOT\_EXIST\\
		Der Konfigurator kann nicht gestartet werde, da der ersten 
		Schritt nicht geladen werden kann. Der Admin des Konfigurators
		muss der ersten Schritt in der Konfiguration erstellen.\\
		Beispielnachricht: Die Konfiguration kann nicht gestartet werden. 
		Bitte wenden Sie sich an der Administrator.
	\item FIRST\_STEP\_EXIST\\
		Der Konfigurator kann ordnungsgemäß gestartet werden.
	\item MULTIPLE\_FIRST\_STEPS\\
		Der Konfigurator kann nicht gestartet werden, da es mehrere
		Schritte erkannt wurde.
		Beispielnachricht: Die Konfiguration kann nicht gestartet werden. 
		Bitte wenden Sie sich an der Administrator.
	\item COMMON\_ERROR\_FIRST\_STEP\\
		wird nicht mehr gebraucht
\end{itemize}

\subsubsection{Common}

\begin{itemize}
	\item SUCCESS\\
		Wenn die Aktion erfolgreich war.
	\item ERROR\\
		Wenn in der Aktion einen Fehler aufgetreten.
	\item CLASS\_CAST\_ERROR\\
		Spezifische Fehler in der DB
	\item ODB\_READE\_ERROR\\
		Spezifische Fehler in der DB
	\item ODB\_CONNECTION\_ERROR\\
		Spezifische Fehler in der DB
\end{itemize}

\subsection{CurrentConfig}

\begin{itemize}
	\item STEP\_CURRENT\_CONFIG\_SUCCESS\\
		???
	\item STEP\_CURRENT\_CONFIG\_BO\_INCLUDE\_NO\_SELECTED\_COMPONENTS
		???
\end{itemize}

\subsubsection{SelectedComponent}
\begin{itemize}
  \item SelectedComponent beschreibt der Status der ausgewhlte Komponente
  \begin{itemize}
    \item REMOVED\_COMPONENT beschreibt der ausgewhlte Komponente, die
    zweites Mal ausgewhlt wurde. Bei ersten Mal wird die Komponente zu der
    Konfiguration hinzugef\"ugt, beim zweiten Mal wird die Komponente aus der
    Konfiguration entfernt. (Abzuwhlen)
    \item ADDED\_COMPONENT beschreibt die Komponente die erstes Mal ausgewhlt
    wurde. Diese Komponente wird zu der Konfiguration hinzugefügt.
    \item ERROR\_COMPONENT beschreibt die Komponente bei dem nach der Auswhl
    einen allgemeinen Fehler passiert.
    \item NOT\_ALLOWED\_COMPONENT beschreibt die Komponente die aufgrund der
    Abh\"angigkeit nicht zu der Konfiguration hinzugef\"ugt werden kann.
  \end{itemize}
  \item SelectionCriterium beschreibt der Status der Komponente vor dem Auswahl.
  \begin{itemize}
    \item siehe Auswahlkriterium
  \end{itemize}
\end{itemize}



\end{document}
