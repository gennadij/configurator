\documentclass{article}

\begin{document}


\section{Title}

\subsection{Subtitle}

\subsection{Logik Beschreibung}

\subsubsection{Dependencies}

Die Abhängigkeiten im Konfigurator bestehen aus folgenden Felder aus:
dependencyType

\begin{itemize} 
  \item exclude
  \begin{itemize}
    \item auto 
    \begin{itemize}
      \item Diese KOmponente und Ihre RequireComponente werder in Konfigurator
      gemerkt und beim Einsteigen in diesem Step werden diese Komponente nicht
      angezeigt
    
    \end{itemize}
    \item marked + selectable
    \begin{itemize}
      \item Die Komponente wird visual merkiert dass sie exclude ist. Nach dem
      Auswahl dieser Komponente bekommt der Benutzer Auswahl zu der möglichen
      Ersatz zu der Out-Komponente, wenn dass nicht m\"oglich wird der Auswahl
      vorzuschlagen wird die Komponente von \"ubergeordnetem Schritt
      vorgeschlagen
    \end{itemize}
    \item marked + unselectable
    \begin{itemize}
      \item Die Komponente wird visuel markiert aber kann nicht ausgew\"ahlt
     werden
    \end{itemize}
  \end{itemize}
\item require
  \begin{itemize}
    \item auto
    \begin{itemize}
      \item Die Komponente wird automatisch in die Konfiguration hinzugef\"ugt.
      Es wird gepr\"uft ob der IN-Komponente weitere require oder exclude
      Komponente hat. Wenn ja, wird gem\"a\ss{} der Abh\"angigkeiten die
      Konfiguration angepasst.
    \end{itemize}
  \end{itemize}
\end{itemize}




\end{document}
