\documentclass{article}

\begin{document}

\begin{titlepage}

\author{Gennadi Heimann} 
\title{Generischer Konfigurator} 
\date{29.12.2017} 
\maketitle
\end{titlepage}

\section{Logik Beschreibung}

\subsection{Abh{\"a}ngigkeiten}

Die Abh\"angigkeiten im Konfigurator bilden die Regel. Jede Komponente hat eine
Ausgangsabh\"angigket (outDependency) und eine Eingangsabh\"angigket
(inDependency). Die Ausgangsabh\"angigket bewirkt auf ihre Komponente mit der
Regel die in der Abh\"angigkeit hintergelegt ist. 
Es gibt zwei Arten (dependencyType) von Abh\"angigkeiten $<exclude>$ und
$<require>$. Die $<exclude>$ Abh\"angigkeit schlie\ss{}t die Komponente aus der
Konfiguration aus. Die $<require>$ Abh\"angigkeit fordert wiederum, dass die
Komponente in die KOnfiguration hinzugef\"ugt werden muss. Das Verhalten der
Abh\"ngigkeiten wird von Parameter $<visualization>$ gesteuert. Der Benutzer
des Konfigurators w\"ahlt die Komponente innerhalb eines Schrittes und die
Logik das Konfigurators pr\"uft die Abh\"angigkeiten und anhand der Parameter
wird weitere Verlauf gesteuert. In der darauffolgende Beschreibung werden
einzelne Parameter erkl\"art.

\begin{itemize} 
    \item visualization bei der $<exclude>$ Komponenten
    \begin{itemize}
        \item auto 
        \begin{itemize}
            \item Die Komponente wird automatisch aus der Konfiguration
            entfernt.Wenn die ausgeschlossene Komponente weiter auf
            eine $<exclude>$ Komponente verwei\ss{}t, werden diese Komponente 
            recursive ausgeschlossen. Das 
            bedeutet, dass bei der sp\"aterem Verlauf werden diese
            Komponente nicht gezeigt. Bei der $<require>$ Komponenten
            wird dieses Vorgehen nicht angewendet. Die Komponente wird nicht aus der 
            Konfiguration entfernt.
        \end{itemize}
        \item marked + selectable
        \begin{itemize}
            \item Die IN-Komponente wird visual merkiert, dass sie
            $<exclude>$ Komponente ist und der Benutzer hat die
            M\"oglichkeit diese Komponente auszuw\"ahlen.
            Wenn der Benutzer w\"ählt diese Komponente aus, wird ein
            Hinweis dem Benutzer vorgeschlagen, wie er die Abh\"angigkeiten
            in seiner Konfiguration l\"osen kann. Es wird eine andere
            OUT-Komponente aus dem gleichem Schritt vorgeschlagen. Wenn in
            diesem Schritt keinen Vorschlag gemacht werden kann wird weiter
            in dem übergeordneten Schritt weiter gesucht. Bei jedem
            vorgeschlagenen Komponente wird weiterhin nach Abh\"angigkeiten
            gepr\"ft.
        \end{itemize}
        \item marked + unselectable
        \begin{itemize}
            \item Die Komponente wird visuel markiert aber kann nicht
            ausgew\"ahlt werden. Die Abh\"angigkeiten werden gleich wie bei der
            Visualizationparameter $<auto>$ behandelt.
        \end{itemize}
    \end{itemize}
    \item require
    \begin{itemize}
        \item auto
        \begin{itemize}
          \item Die Komponente wird automatisch in die Konfiguration hinzugef\"ugt.
          Es wird gepr\"uft, ob der IN-Komponente der Abh\"angigkeit weitere
          $<require>$ oder $<exclude>$ Komponente hat. Wenn ja, wird gem\"a\ss{} der
          Parameter in der Abh\"angigkeit die Konfiguration angepasst. Wenn die
          IN-Koponente über mehrere Schritt in der Konfiguration entfernt ist. Der
          Konfigurator wird, dann in Laufe der Konfiguration bei jedem Schritt dem
          Benutzer einen Hinweis geben, welche Komponente der Benutzer ausw\"ahlen
          kann/muss um zu dem Schritt mit der $<require>$ Komponente zu kommen. Die
          Komponente wird visual markiert, dass sie schon zu der Konfiguration
          hinzugefugt wurde. Diese Komponente ist auch $<unselectable>$. Bei der
          Abh\"angigkeit in gleichem Schritt wird die Komponente ohne jegliche
          Hinweise zu der Konfiguration hinzugef\"ugt und dementsprechend markiert.
        \end{itemize}
    
        \item selectable
    
        \begin{itemize}
            \item Bevor die Komponente zu der Konfiguration hinzugef\"ugt wird, wird
            dem Benutzer einen Hinweis mit dem Auswahl gegeben. Der Benutzer kann
            entscheiden ob er die required Komponente zu der Konfiguration
            hinzuf\"ugen will.
        \end{itemize}
    \end{itemize}
\end{itemize}




\end{document}
